\section{Some More Tips}
\textbf{Title Capitalization: } \\ There are \href{https://en.wikipedia.org/wiki/Letter_case#Stylistic_or_specialised_usage}{two main options} for title capitalization: either title case, where you capitalize every significant word, or sentence case, where you capitalize mostly only the first word. Choose one and stay consistent with your titles. \\\\
\textbf{Word Count:} \\
For the seminar papers, you should write about 3500 to 4000 words. If you are using Overleaf, there is an integrated word counter. You can find it, if you go to \texttt{Menu} on the upper left side and then \texttt{Word Count}. In this document, the parts that should not be included for counting are already marked with \texttt{\%TC:ignore} at the beginning and \texttt{\%TC:endignore} at the end. \\\\
\textbf{Literature Management:} I will just give a hint that a literature management tool, such as \href{https://www.zotero.org/}{Zotero} can be very helpful for a seminar paper like this but also for a future thesis. This allows you to have a place where all your literature is saved. You can annotate within the program, create folders, tags, notes, etc. There is even a plugin for most browsers, which allows you to include it directly from where you found it (pay attention to the author, year, conference information here as well!). Zotero is free and open source, only the synchronization of the files between different devices is not unlimited. \\\\
\textbf{Overleaf Professional:} The professional version of Overleaf (online LaTeX editor) is covered by the \href{https://de.overleaf.com/edu/tuw}{TU Wien}! Most things also work with the basic functionality, but it can be a nice add-on to have the professional version. 
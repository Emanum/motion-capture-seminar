% This is a template for seminar papers in the courses by Prof. Margrit Gelautz. There are important informations within this document, so please have a look at it at the beginning thorougly! When "TODO" is written, it means that there are changes you need to do here. When "COMMENT" is written, it means that we are trying to explain the LaTeX code for you.

\documentclass[titlepage, 12pt]{article} % The document gets initialized.
\newcommand{\open}[1]{\textcolor{red}{#1}} % When you write: \open{} the text in brackets is red.

% COMMENT: Useful packages
\usepackage[english]{babel} % Language setting
\usepackage[a4paper,top=3cm,bottom=3cm,left=4cm,right=4cm,marginparwidth=1.75cm]{geometry} % Set page size and margins
\usepackage{amsmath, graphicx, gobble, array, xcolor, soul}
\usepackage[colorlinks=true, allcolors=blue]{hyperref} % Creates colored links
\usepackage{tabularx}
\usepackage{pdflscape}

\setlength\parindent{0pt} % This removes the indents

% TODO: Please fill in your information here
\newcommand{\paperTitle}{Cost Effective Real Time Motion Capture for 3D Avatars}
\newcommand{\studentName}{Manuel Eiweck}
\newcommand{\matriculationNumber}{01633012}
\newcommand{\studyCode}{UE 066 932}
\newcommand{\studentEmail}{manuel.eiweck@tuwien.ac.at}
\newcommand{\seminarNumber}{193.179}
\newcommand{\seminarName}{Seminar in Visual Computing}
\newcommand{\supervisor}{Prof. Dr. Margrit Gelautz}
\newcommand{\semester}{2025W}
\newcommand{\dueDate}{Draft: 7.12.2025 \\ Final: 5.1.2026}

% COMMENT: This is where your title page is formatted, don't change anything here.
\title{
\textbf{\paperTitle}}
\author{\\ 
    \textbf{\studentName} \\ \\ \\ 
    Matriculation Number: \matriculationNumber \\ 
    Study Code: \studyCode \\ 
    E-mail: \studentEmail \\ \\ 
    \seminarNumber \\
    \seminarName \\
    Supervisor(s): \supervisor \\
    Semester: \semester \\
    \newline
}
\date{\dueDate}

%TC:ignore
% #########################################
% COMMENT: The actual document and content starts here. 
\begin{document}
\maketitle

% #########################################
% COMMENT: The declaration of independent work is inserted here. Please read it again.
\section*{Erklärung zur Verfassung der Arbeit}
Hiermit erkläre ich, dass ich diese Arbeit selbständig verfasst habe, dass ich die verwendeten Quellen und Hilfsmittel vollständig angegeben habe und dass ich die Stellen der Arbeit - einschließlich Tabellen, Karten und Abbildungen -, die anderen Werken oder dem Internet im Wortlaut oder dem Sinn nach entnommen sind, auf jeden Fall unter Angabe der Quelle als Entlehnung kenntlich gemacht habe. \\
Ich erkläre weiters, dass ich mich generativer KI-Tools lediglich als Hilfmittel bedient habe und in der vorliegenden Arbeit meine gestalterischer Einfluss überwiegt. Im Anhang “Übersicht verwendeter Hilfsmittel” habe ich alle generativen KI-Tools gelistet, die verwendet wurden, und angegeben, wo und wie sie verwendet wurden. Für Textpassagen, die ohne substantielle Änderung übernommen wurden, habe ich jeweils die von mir formulierten Eingaben (Prompts) und die verwendete IT-Anwendung mit ihrem Produktnamen und Versionsnummer/Datum angegeben. \\\\
% \open{IMPORTANT: Read this thoroughly so that you make no mistakes here! (it is in German for legal reasons, but you'll find a way to translate it)} % TODO: Remove this before the final submission.

% #########################################
% TODO: Here you should write your abstract.

\newpage
\addcontentsline{toc}{section}{Abstract} % This adds Abstract to the Table of Contents. 
\section*{Abstract}

3D Motion capture for the professional industry is traditionally being done with a marker based approach in complex studio setup including multiple cameras and special suits to achieve high accuracy.
While this might suit certain needs such as film production and offline computer animation, other use-cases require real-time processing and high mobility instead of high accuracy.
Furthermore, expensive investments in a motion capture studio create a high entry barrier for smaller organizations and individual creators. 
Therefore, this seminar papers tries to find a meaningful balance between high-cost high accuracy setups and low cost high mobility setups.\\\\
Human motion capture covers a large research field, various kinds of input data, output data a lack of common terminology and standards make it hard to compare approaches. 
To give an overview of the possibilities in motion capture methods 5 different methods are examined in detail.
Benchmark datasets are presented alongside different output file formats and evaluation metrics.
Afterwards, a use-case for 3D Live concerts using virtual avatars (VTuber) is described, and motion capture tasks are formulated. 
With a certain use-case and various kinds of approaches in mind a qualitative comparison and quantitative comparison of 8 different methods is done. 
Lastly, a potentially suitable motion capture method is described, and current challenges are explained.




% Motion capture technology has traditionally relied on expensive multi-camera marker-based systems that require fixed studio setups and significant infrastructure investment. However, recent advances in deep learning and computer vision have enabled cost-effective alternatives suitable for smaller organizations, individual creators, and applications requiring portability. This seminar paper provides a comprehensive overview of current state-of-the-art motion capture methods that balance accuracy, cost, and real-time performance for practical applications such as virtual avatar animation in live performances.

% We examine five prominent approaches representing different sensor modalities and architectures: OpenPose, a widely-adopted real-time 2D multi-person pose estimator; DeMoCap, which uses low-cost depth sensors for marker-based 3D tracking; XNect, enabling markerless 3D multi-person tracking from a single RGB camera; Ultra Inertial Poser, combining IMU sensors with ultra-wideband technology for wearable-based tracking; and MAMMA, a recent method achieving near gold-standard accuracy with markerless multi-camera setups. Additionally, we discuss relevant datasets, data formats, and industry standards that support these approaches. We then evaluate these methods against specific requirements for real-time motion capture in live concerts with virtual avatars.

% While no single existing approach fully satisfies all requirements for this use case, recent developments demonstrate significant improvements toward making motion capture more accessible to broader audiences. The trade-offs between accuracy, cost, portability, and real-time performance vary across methods, making the selection of appropriate techniques dependent on specific application constraints and intended use cases.

% #########################################
% COMMENT: This is where your table of content is automatically generated.

\newpage
\tableofcontents
%TC:endignore

% #########################################
% COMMENT: Here, your real content of the seminar paper starts.

\newpage
\pagenumbering{arabic}

% COMMENT: We recommend to create separate files for the separate sections. It simplifies the main.tex file and makes it easier to keep a good overview. You can use \input to include the content of the other file on the same page. Or you can use \include if you want to start a new page with the content.

% \section*{Introduction to this Document}
% Hello all. This document is supposed to offer you some insights into LaTeX, scientific writing, and give some tips for writing your seminar paper. The content is specifically catered for this seminar and not just generic LaTeX information. If you are looking for a more basic introduction, you can have a look at this \href{https://www.overleaf.com/learn/latex/Learn_LaTeX_in_30_minutes}{LaTeX tutorial}. Please read through the \textbf{whole document thoroughly}, because it focuses on typical mistakes of previous semesters. Also read through the \textbf{introductory slides} again. Please also note: the following is based on my personal experience, can contain mistakes or accidental misinformation.
% \\\\
% Concerning the template, you don't need to use this template specifically. You are also free to use a different one, but then pay attention that you include all the important aspects which are mentioned in the preliminary slides.

\pagebreak


\section{Introduction}
% Use the introduction to build a clear storyline that highlights the importance of the topic and engages the reader. Also, give a short overview of what the reader can expect. You should have a lot of references here, as you are laying the scientific basis for the rest of the paper.

% \hfill

% While we see professional VTuber Concerts being popular in Asia, we also saw some increasing popularity in the west as well. 
% While big studios use professional grade motion capture systems, similar to these used in the movie business, there are also smaller groups. 
% We believe that with the recent trends and methods of deep learning motion capture systems, more compact systems that require less camera are capable of achieving similar results which are good enough for these use cases. 

% \hfill

Human motion capture is heavenly used in the movie industry for computer generated imagery (CGI), dating back all the way to an animation technique called rotorscoping in 1915 by the cartoonist Max Fleischer. While the implementation of these techniques got modernized from a projected movie on a piece of paper to optical sensor based systems, the goal stayed the same: Capturing real world human motion data and translating them into a virtual representation for further processing such as in traditional 2D hand-drawn animations or modern 3D computer generated animations \cite{menache_understanding_2000}. \\
To achieve as much accuracy as possible, traditional state-of-the-art motion capturing studios use a large quantity of high resolution and high frame rate camera systems in a multi view setup combined. This is often referred to as the "gold standard" in literature.
Professional studio equipment as well as proprietary software is developed by companies like Vicon \cite{VICON}, Optitrack \cite{optitrack} or Qualisys \cite{qualisys}.
While smaller areas can be captures using only 4 cameras \cite{sigal2009humaneva-8bd}. Cover corporation demonstrate the scaling possibilities of this technique by equipping a 23 m x 14 m x 3.2 m studio with 200 Vicon Valkyrie (VK-26) \cite{hololive_setup}. 
Similar setups are also applied in the research area of human pose estimation when high accuracy is requires such as the generation of ground truth datasets. HumanEva \cite{sigal2009humaneva-8bd} build their capture setup using 4 cameras from Vicon MX. MoVi \cite{10.1371/journal.pone.0253157} combines 15 stationary Qualisys cameras and 17 IMU sensors.
More modern datasets targeted towards deep learning approaches such as the HumanOLAT \cite{teufelgera2025HumanOLAT} use 40 RED Komodo 6K cameras and 331 individually controllable LEDs for a full 360 degree capture dome. The CMU Panoptic \cite{Joo_2017_TPAMI} uses over 500 cameras together with other sensors.\\
Approaches following the "gold standard" usually apply fundamentally computer vision algorithms that are already well studied. 
A first step is getting marker positions which can be done with model-based, region-based or feature-based methods \cite{wang_recent_2003}. Afterwards, multi-view geometry is applied to do 3D pose estimation. 
During this step triangulation is used to calculate the 3D position based on the matching 2D positions from the tracked points as well as intrinsic and extrinsic camera parameters \cite{desmarais_review_2021}.
One key challenge in motion capturing is occlusion. A simple yet effective solution to deal with this is eliminating as much of it in the first place by using multiple cameras. 
Therefore, many state-of-the-art motion capture systems in the industry as well approaches in research use a large amount of cameras to cover as many angles as possible.

\hfill 

While these methods still deliver accurate results, they also have some significant downsides.
One issue is the requirement for a fixed physical capture area. Recent trends in virtual reality (VR) and augmented reality (AR) created the need for egocentric motion capture systems based on the camera perspective available in head-mounted devices (HMDs). The presence of dynamic moving environments and therefore missing camera calibration making these tasks especially challenging. FLAG \cite{FLAG} and EgoPoser \cite{jiang2024egoposer} apply a transformer based deep learning approach to approach this problem.\\
Another obvious issue is the high hardware cost due to the sheer amount of high resolution camera being used. Increased costs and unflexible setups might not be an issue for professional motion capture studios or the dataset generation in research projects. However, it creates a high entry barrier into motion capture for smaller organizations and individual creatives such as 3D artists, 3D animators or VR creators and musicians performing via a virtual avatar (VTuber). 
Modern approaches often require a significantly lower number of cameras. Additional sensor data such as IMU sensors, deep learning and specialized techniques such as modelling of limbs is used instead \cite{chatzitofis2021democap} \cite{mehta_xnect_2020} \cite{openPose} \cite{VIBE} \cite{simpoe} \cite{leonardis_avatarpose_2025}
\cite{FromMethodstoApplicationsAReviewofDeep3DHumanMotionCapture} \cite{McInIndustryASystematicReview}.

\hfill

% While professional multi-view motion capture systems will continue to provide the most accurate motion capture results,
% can fill in the gap between high accurate multi-view motion capture systems and 

We believe that recent developments in the 3D pose estimation research field can provide motion capture systems for individual creators or smaller companies. 
The focus of these systems target high portability, lower costs, less invasive to the human body and application in dynamic environments compared to traditional systems installed in permanent motion capture studios. We want to explore these recent methods and technologies and evaluate their applicability for these use cases.

\hfill 

The motion capture research field is large and there exist many completely different approaches using different sensors, architectures, requirements and goals in mind. Therefore, in chapter \ref{chapter_methods} we are presenting some popular and current state-of-the-art methods that all approach these challenges using different techniques.
Chapter \ref{chapter_tech} describes datasets, data format as well as protocol and standards used in the industry. 
In \ref{chapter_task} our task is presented which focus around real time motion capture for live concerts with virtual avatars. 
In \ref{chapter_discussion} we categorize and compare approaches. 

% Note to myself:
% DeMoCap \cite{chatzitofis2021democap} has some good infi for intro


% Animate virtual Avatars 

% Optical human motion capture is a key enabling technology in
% visual computing and related fields [Chai and Hodgins 2005; Men-
% ache 2010; Starck and Hilton 2007]. For instance, it is widely used
% to animate virtual avatars and humans in VFX. It is a key compo-
% nent of many man-machine interfaces and is central to biomedical
% motion analysis
% FROM XNECT


% gold-standard
% marker-based solution


% Sony Mocopi case study-> example Usage
% https://pro.sony/ue_US/press/uc-san-diego-spatial-reality-display-mocopi-case-study

% Commercial systems
% https://www.motionanalysis.com/
% https://www.optitrack.com/
% https://www.qualisys.com/

% Company ffrom the max planck institute guy
% https://meshcapade.com/


% AMASS: Archive of Motion Capture as Surface Shape -> discusses why gold standard is ass

% While
% commercial markerless systems exist [Captury 2025; Move AI Ltd.
% 2025; Moverse 2025; Theia Markerless 2025], they remain expensive
% and are generally considered to be less accurate than marker-based
% systems (the “gold standard”)

\section{Methods}
\label{chapter_methods}
Keywords:
 monocular video, kinematics, global coordinates, dynamic cameras,  infills missing poses,

\subsection{Non Real Time Approaches}
PACE: Human and Camera Motion Estimation from in-the-wild Videos \cite{PACE}

\subsection{Not categorized yet}
SimPoE: Simulated Character Control for 3D Human Pose Estimation \cite{simpoe}

VIBE: Video Inference for Human Body Pose and Shape Estimation \cite{VIBE}

Avatarpose \cite{leonardis_avatarpose_2025}


\pagebreak

\subsection{OpenPose - marker less 2D}

OpenPose is a popular open-source framework for real-time 2D multi-person and 3D single-person pose estimation with over 30K stars on GitHub.
There are multiple publications related to OpenPose. Convolutional Pose Machines \cite{openPose_4_wei2016cpm} started with a sequential architecture of convolutional neural networks that produces 2D belief map.
Each network describes a stage, the output of one stage is used as input for the next stage. By using a fully differentiable model they can use backpropagation for training. One issue with their approach is handling multiple people in the same image.\\
This is addressed in their next publication with Part Affinity Fields \cite{openPose_3_cao2017realtime}. A bottom-up instead of top-down approach is used. This means that first body parts are detected and afterwards assigned to an unknown number of people in the image.
They extend their sequential architecture with a two-branch approach. One branch predicts part affinity fields which encode position and orientation of limbs in a 2D vector field, another the confidence maps of body parts at a certain location. By using techniques from graph theory, they can match these body parts to the vector field and also combine them into a full body pose. With their architecture and greedy matching algorithm, they can achieve real-time multi-person 2D pose estimation. \\
In \cite{openPose_2_simon2017hand}, they propose a method that produces 3D motion capture results handling complex Occlusion scenarios. They build up on their previous models to predict 2D estimation via multiple camera angles. The result is then triangulated to get 3D results. 
However, compared to other approaches they go a step further and use these results as their training data together with their 2D estimations for a model which enables markerless 3D motion capture outputs on hands from single view RGB images.
In \cite{openPose}, they extend their 2D multi-person pose estimation to also include foot keypoints and facial landmarks, compared to their previous wor they only use Part Affinity Fields. Furthermore, this work also does not include any 3D motion capture cababilities. In their GitHub \cite{openPose_github} they provide an option to detect 3D keypoints including face, hand and foot features using multiple cameras. However, as it uses simple triangulation it has the limitation of a fixed camera setup and only works with a single person in the scene.

\subsection{DeMoCap - marker based 3D} 
Chatzitofis et al. proposed their 3D marker based motion capture system DeMoCap \cite{chatzitofis2021democap}. Their work includes the method itself, focused on providing a low-cost alternative to the classical optical marker based solutions by using consumer-grade infrared-depth camera only. In addition, they release the dataset used for training, which contains colored infrared and depth images with 3D pose and marker annotations in multiple views. The ground truth data comes from professional motion capture system with 24 cameras while the depth data comes from 4 stereo based depth sensors.\\
Their proposed method works as follows: First for each of the multi-view depth images 3D positions of the markers are extracted and normalized. This work as each marker reflects the infrared rays with a different intensity therefore identifying the marker. 
These markers together with joint heatmaps are fed into a fully convolutional network that transforms the markers to poses. This is done via their own 3D regression model. \\
While their methods performs quiet good even under complex scenarios on public datasets such as the SFU Dataset \cite{sfudataset} it comes with some serious limitations. 
Their used depth sensors have a distance limitation of 4 meters, which limits the capture space significantly especially compared to optical sensor based systems. Furthermore, the quantity of fast poses are challenging due to the low 30hz frequency of the sensor. 
With only 2-4 cameras required the system is fairly mobile and quickly to set up. However, it requires the actor to wear a special suit with 53 placed markers which make it far from ideal for applications in the wild.

\subsection{XNect - marker less 3D}

Xnect \cite{mehta_xnect_2020} is a real-time marker less approach that is powered by a single monocular RGB based camera and can provide temporally coherent tracing tracking for multiple people in diverse scenes in the wild.

Their bottom up architecture start by applying a convolutional neural network which is trained to detect only fully visible features such as the joint itself, or it's parent/child. The networks output are 2D body joint heatmaps, part affinity fields to assign joints to individual persons as well as their own 3D pose encoder which uses local pixel information close to the 2D joint position. 
In a second step a lightweight fully connected neural network transform the 2D features and 3D encoder to 3D poses for each person. In addition, it uses previous learned 3D poses to handle occlusion within the scene. 
Lastly, a space-time skeletal model is applied to further improve the 3D pose. 
The result or the last two stages is used in combination with the results of previous frames to fit the 3D pose to a kinematic skeleton and therefore enforce temporal coherence.
The end result of the pipeline is a full skeletal pose with joint angles for each person.

While their CNN applies a novel CNN architecture called SelecSLS to achieve real-time performance, they mention that any CNN architecture for keypoint prediction can be used.
This makes their approach potentially be extended in the future by upcoming CNN network architectures. 

They compare their approach to Microsoft Kinect V2 which uses depth sensing cameras with similar and sometimes better qualitative performance especially for scenes including occlusions. 
In addition, results on common datasets such as MuPoTS-3D benchmark dataset \cite{singleshotmultiperson2018} and the Panoptic dataset \cite{Joo_2017_TPAMI} \cite{Joo_2015_ICCV} are presented and compared to SingleShot \cite{singleshotmultiperson2018}, LCRNet \cite{shi2024tro}, MP3D \cite{mp3d}, LCRNet++ \cite{lcrnetplusplus} and PoseNet \cite{resnet}. The only approach that had a better score was PoseNet, however it comes with other limitations one of which is the lack of real-time performance. Their own CNN SelecSLS is compared to ResNet \cite{resnet} with similar results while still also lacking real-time cababilities.
However, compared to multi-view capture systems XNect has far less accuracy. 
While occulusion is partly handled via their 2nd stage, it is far from perfect. 
Occlusion of the neck of a person result in the entire person not being detected regarding if large parts of the body is detected. Close human interactions like hugging also leads to inaccurate results. 
Using previous frames to match a skeleton can lean to issues for the first frames of occlusions. The simple person tracker also lead to mismatches between the capture and therefore a swap.
% https://universaar.uni-saarland.de/bitstream/20.500.11880/29908/1/thesis.pdf

\subsection{Ultra Inertial Poser - IMU based 3D} 

Tracking systems using IMU sensor are not new. We have seen various approaches like TransPose \cite{TransPose} and EgoLocate \cite{EgoLocate2023} as well as commercial products like Sony Mocopi \cite{mocopi}. One issue with IMU sensors is that they are prone to drift and jitter and therefore are not as reliable as other tracking methods. 
Ultra Inertial Poser \cite{UltraInertialPoser} attempt to increase tracking accuracy and reliability by combining the IMU data with ultra-wideband (UWB) ranging sensor data. 
Their systems work with six embedded trackers each combing with a 6-DoF IMU and a UWB sensor plus emitter to measure inter-sensor distances.
UWB signals use a large bandwidth together with short waveforms, based on that physical property there exist multiple protocols used in signal processing to measure distance: Two-Way Ranging, Time Difference of Arrival and Angle of Arrival.
One challenge with UWB is to handle noise, which arises when objects, such as body joints, prevent a direct line of sight between sensors. Ultra Inertial Poser uses an Extended Kalman Filter to reduce the noise.

Each tracker is equipped with an integrated microcontroller which samples raw IMU data and an estimated distance to other trackers. The raw data is streamed via Bluetooth low energy to a host computer which performs the rest of their tracking pipeline.
First a LSTM network together with a distance attention graph convolutional network (DA-GCN) is used to determine the positions of the trackers in 3D space.
Afterwards a kinematics decoder based on the work Physical Inertial Poser (PIP) \cite{PIPCVPR2022} is used to estimate the final SMPL pose parameters as well as a global translation relative to an initial T-Pose calibration performed by the user.

Their evaluation shows that their approach with the combination of IMU and UWB data for tracking performs better than other IMU only based methods like PIP \cite{PIPCVPR2022}.
They also confirmed that the IMU sensor quality highly influences drift errors. As other methods performed way worse on their dataset gathered with inexpensive IMU than synthetic datasets or data gathered from high-end sensors.
The difference was bigger for small acceleration motions as pure IMU methods tend to perfom worse here. Another advantage of the inclusion of UWB distance data is that pure IMU methods tend to mistake sitting poses with standing poses where a combined data approach can correctly classify those cases. Furthermore, their LSTM and DA-GCN network were able to reduce jitter error.

However, they do not compare their approach against camera based tracking methods, which are way less prone to jitter in general. A video presentation \cite{UltraInertialPoser_YT} of their work shows a clear improvement compared to pure IMU based methods. Nevertheless, compared to the ground truth there is a still a large jitter movement visible to the naked eye.


% Issue: UWB sensitive to temperature
% Issues: missing Real world, drift, hard poses not captured as no training data


\subsection{MAMMA}

Traditional "gold standard" motion capture studio use marker based methods to produce sub-millimeter accurate tracking. MAMMA (Markerless and Accurate Multi-person Motion Automatic capture) \cite{cuevas2025mamma} claim to achieve similar results with a markerless approach in a studio setup including 32 RGB cameras.
Their algorithm only uses multiple multi view videos as input and estimate a dense 2D surface for multiple people in complex scenarios. The motion capture data is predicted in the SMPL-X format \cite{SMPLX} and includes hand gestures. \\
A network consisting of vision transformer (ViT) and a CNN is trained to extract images features. They use a technique called landmark queries, which uses multiple embeddings, to further increase generalization of the network. This increases accuracy for challenging poses that are not present in the training data. Multiple embeddings are possible because their technique can use the entire video input compared to just relying on marker positions.
The detected features of the ViT and CNN are combined and used first in a Transformer Decoder and secondly in an MLP network to predict 2D landmarks for each camera view. These landmarks include position via pixel coordinates as well as a numerical value stating the probability of the feature being visible on camera. 
In The final stage these landmarks are fit to a SMPL-X 3D body including multiple stages of optimization.\\
They train their network on a pure synthetic dataset which simulating their 32 camera studio setup, consisting of single person scenes, complex interaction scenes including dance scenes with two persons and a subset focusing on hand articulations. \\
An evaluation on the Hi4D dataset \cite{yin2023hi4d} show that their method outperform other approaches. The best in class, Avatarpose \cite{leonardis_avatarpose_2025} was beaten in tracking accuracy with a value of 13.43 to 32.10 Mean Per-Joint Position Error (mm).
Furthermore, a direct comparison to Vicon \cite{VICON} is performed using a real evaluation dataset. Results show that there is only a minimal 1.611mm difference compared to the "gold standard". This make their technique accurate enough for many applications. An interesting comparison is also done in terms of usability and cost. Capturing a dataset with Vicon requires manual post-processing. 3 skilled technicians worked in total close to 47 hours to perform the cleanup of a 24-minute scene. Computation for the final SMPL-X format took another 25 hours. MAMMA does not require manual post-processing and took about 8 hours on an RTX-4090 to compute the final motion tracking result.\\
As for limitations heavenly occluded areas are still a problem and can cause flickering or over-smoothing. In addition, the accuracy on hand tracking is not ideal. Compared to other methods it is also not useable in real-time scenarios and a fixed studio setup with a larger amount of camera is required. 
\section{Motion Capture in Industry}
\label{chapter_tech}
An applied motion capture systems can be seen as a pipeline with various software modules in between. A transformation from the real human motion through tracking and processing to the final result like a rendered 3D video, animations in a game engine, motion data as an HCI device or other real time applications is done.
In a quickly evolving field like motion capture it is especially important to have common file formats, standards and protocol to exchange the tracking information in order to allow methods to focus on a single task.
Standardized interfaces allow us to develop a modular software architecture and upgrade our tracking methods later down the line without having to adapt our entire motion capture to render pipeline. 
This allows commercial and open source integrations into other software to be developed. \\
Furthermore, we need benchmark datasets to measure and compare different approaches quantitatively. This allows appropriated methods to be picked based on a certain use case. In addition, it enables accurate evaluations for new approaches.

\subsection{Tracking Format}

A simple solution to exchange tracking information is to store it as a set of certain keypoints in 2D or 3D coordinates and then serialize it either as plain text or in binary file format. 
Various approaches use different body locations for their keypoints often as a result of different tracking methods. For marker based approaches these are usually equal to the marker position attached on the body.
While there are no defined standards in the research community we have seen approaches using similar keypoint position as common datasets are using. This can be observed in the dataset HDM05 \cite{cg-2007-2} and COCO Pose \cite{coco} with their usage in OpenPose \cite{openPose_docs}. However, OpenPose also proposed their own skeleton modal body25.\\
A common proprietary format used in the 3D graphics industry is FBX \cite{fbx_format}. It contains alongside mesh information, material and texture also skeletal animation information that can be used to store motion capture data. 
Other similar multi asset format are Collada \cite{Collada_format} originally developed by Sony or the more recent GLTF \cite{gltf_format} both are now maintained by the Khronos Group.
While these formats have support for meshes, they are targeted towards general 3D models and are therefore not optimized for storing human bodies.
A mesh and skeleton based storage format limits the capabilities of some approaches. In order to predict entire body shapes and the movement of skin and fine facial details either a new mesh would need to be stored for each key frame or an increase of the skeleton count would be required. \\
SMPL \cite{loper_smpl_2015} improves the storage options for such approaches. It is a format specialized to capture human skinned models. Instead of storing a complex mesh for the entire body SMPL defines a base mesh and skinning equations to further refine the mesh.
It formulates a function which takes tracking parameters along with a pre learned model as input and output a final vertex mesh for each key frame.
SMPL comes with the weights for the model which is trained by analyzing thousands of body scans.  
% Therefore, trading runtime speed against storage capacity. 
SMPL-X \cite{SMPLX} improves the original model by adding tracking for hands and expressive face details as well as using an updated dataset for training.
Both offer along the actual storage and model mechanism tools to calculate a final mesh. This is done via vertex based linear blend skinning with learned corrective blend shapes.\\


VRM File Format \cite{VRM_consortium}
VRM Press Release \cite{vrm_press_release}

VMC Protocol Specification \cite{vmc_protocol_specification}


Sony Mocap Protocol https://www.sony.co.jp/en/Products/mocopi-dev/en/documents/Home/TechSpec.html

Webcam Motion Capture + Sony Mocap
https://www.sony.co.jp/en/Products/mocopi-dev/en/documents/Home/TechSpec.html


\subsection{Datasets}
AMASS: Archive of Motion Capture as Surface Shapes \cite{AMASS:ICCV:2019}

CMU Panoptic Studio datase

Human3.6M, HumanEva, and MPI-INF-3DHP dataset


DanceDB (included in AMASS)

UnrealEgo Akada et Al 2022\\

\section{Problem Statement}
\label{chapter_task}

Describe the Task that needs to be solved in general. 
Realtime, Occlusion, Low cost


\section{Discussion}
\label{chapter_discussion}

While you don't have to include a discussion section, it is encouraged that you try to add to existing knowledge in some way. This may be through comparison of existing methods, critical reflection, etc.

% Modal based constraints: 


% we fit a model-based skeleton to the 3D and 2D predictions in order
% to satisfy kinematic constraints and reconcile the 2D and 3D predic-
% tions across time -> XNect


% Below is a table positioned exactly here:\\

\noindent\makebox[\textwidth]{%
\begin{tabularx}{1.4\textwidth} { 
  | >{\raggedright\arraybackslash}X 
  | >{\raggedright\arraybackslash}X 
  | >{\raggedright\arraybackslash}X 
  | >{\raggedright\arraybackslash}X 
  | >{\raggedright\arraybackslash}X | }
 \hline
 \textbf{Approach} & \textbf{Sensor Setup} & \textbf{Output} & \textbf{Time complexity} & \textbf{Key Limitations} \\
 \hline
 \textbf{OpenPose} 2021 \cite{openPose} (\ref{method_openpose}) & markerless \newline single RGB camera for 2D \break multi-view RGB cameras for 3D & multi person 2D keypoints including foot and hands. Option for single person 3D keypoints via triangulation in a multi-view setup & Realtime & 3D capabilities very limited; Body occlusion and high crowded images lead to false keypoints. \\
 \hline
 \textbf{DeMoCap} 2021 \cite{chatzitofis2021democap} (\ref{method_democap}) & 53 body markers \newline  + 4 low-cost depth Cameras  & single person 3D Pose in SMPL & Realtime & limited capture space due to 4m range limit of sensors \newline required 53 markers or retraining \\
 \hline
 \textbf{XNect} 2020 \cite{mehta_xnect_2020} (\ref{method_xnect}) & single RGB Camera & multi person 3D Pose as Skeleton & Realtime & Occlusion of neck leads to complete loss of person; Issues with complex interactions; person mismatch between frames \\
 \hline
 \textbf{Ultra Inertial Poser} 2024 \cite{UltraInertialPoser} (\ref{method_uip}) &  6 wearable trackers with IMU + UWB & single person 3D Pose in SMPL & Realtime & clear visible jitter artifacts; low tracking position accuracy (order of 5-11 cm)  \\
 \hline
 \textbf{MAMMA} 2025 \cite{cuevas2025mamma} (\ref{method_mamma}) & 33 high quality RGB Cameras & multi person 3D Pose in SMPL-X & Offline - processing ~ 3x than recording & Offline processing, high number of camera required, low mobility \\
 \hline
 \textbf{SimPoE} 2021 \cite{simpoe} & single RGB Camera & single person 3D Pose (compatible with body mesh models like SMPL)& & \\
 \hline
 \textbf{VIBE} \cite{VIBE} & & & & \\
 \hline
 \textbf{Avatarpose} \cite{leonardis_avatarpose_2025} & & & & \\
 \hline
 \textbf{PACE} \cite{PACE} & & & & \\
 \hline
 \textbf{FLAG} \cite{FLAG} & & & & \\
 \hline
 \textbf{EgoPoser} \cite{jiang2024egoposer} & & & & \\
 \hline
 \textbf{TransPose} \cite{TransPose} & & & & \\
 \hline
 \textbf{EgoLocate} \cite{EgoLocate2023} & & & & \\
 \hline
 \textbf{Sony Mocopi} \cite{mocopi} & & & & \\
 \hline
 \textbf{Physical Inertial Poser (PIP)} \cite{PIPCVPR2022} & & & & \\
\hline
\end{tabularx}%
}
\pagebreak
\section{Conclusion}

Largely different approaches to motion capture including various sensor data, different goals in mind and non standardized datasets makes the research field of motion capture a complex topic to get started with.
In this seminar paper an overview of the current state-of-the-art for motion capture was given by examining a broad spectrum of different approaches, categorizing and comparing them accordingly.
All state-of-the-art approaches using data driven workflows, this means that getting good training data is nearly as important as the algorithm and model itself.
The use-case of Vtuber live concerts was examined, specific goals and tasks extracted and possible solution was drafted.
An existing approach that fulfills all goals completely was not found. However, great improvement has been made by various researchers and people within the industry in the last years to achieve impressive results. 

%TC:ignore
% \section{Short Introduction into LaTeX}
Always include a small introductory section before you start the next subsection. Also, if you only have one subsection, leave it out. In this case: In this section, you can find information about figures, BibTeX entries (including databases, bibliography style, and citations), and common mistakes to avoid.

\subsection{Inserting Figures}
You can and should add figures to your seminar paper. This works using  \textit{\textbackslash begin\{figure\}}. There are five things to look out for here:
\begin{enumerate}
    \item You can add a [h!] after \texttt{\textbackslash begin\{figure\}}. This helps to position the figure in the text, where you mentioned it (though this is not 100\%).
    \item You should give a meaningful and detailed description of the figure in the caption. You can do this in the \texttt{caption} field of the \texttt{\textbackslash begin\{figure\}} (see \autoref{fig:color-blind-chart} for an example).
    \item Make sure that you reference where you have the image from. If you created it yourself, you can but don't necessarily need to mention this. If you adapted it from somewhere, note that as well.
    \item Your figure should be readable and accessible. This means that the text in the figure/image should be approximately the same font size as the rest of your text. Also, have a look for the colors, they should be readable for color-blind folks as well. I can recommend this tool to check for that: \href{https://www.color-blindness.com/coblis-color-blindness-simulator/}{Color Blindness Simulator}.\\
    \begin{figure}[h!]
        \centering
        \includegraphics[width=\linewidth]{graphics/Color_blindness_wheels.png}
        \caption{Three different color blindness types are presented with the standard vision (Trichromacy) in comparison. These include Deuteranopia (green-blind), Protanopia (red-blind), and Trianopia (blue-blind) (adapted from \cite{Hawesthoughts2023Colorblindness}).}
        \label{fig:color-blind-chart}
    \end{figure}
    \item And lastly, reference the figure in the text using \texttt{ref\{\}}. Ideally, you discuss the figure in the surrounding text to integrate it in the structure and justify its inclusion in the seminar paper. Like so: \textit{A visualization of common color blindness types can be found in Figure~\ref{fig:color-blind-chart}. Most common is Deuteranopia, which makes the distinction between red and green especially difficult (compare to Figure \ref{fig:color-blind-chart}).}
    You can also use \texttt{autoref\{\}} which look like this: \autoref{fig:color-blind-chart}. 
\end{enumerate}

\subsection{Citations in LaTeX}
If you've worked with LaTeX before, you are probably familiar with the following information. In LaTeX, it is common that you have a separate bib-file (here: \texttt{references.bib}), which contains your BibTeX entries. A BibTeX entry contains all relevant information about a reference. There are different entry-types, such as \texttt{article}, \texttt{book}, etc. In the following, you will get information about sourcing BibTeX entries, inserting them into your project, choosing a bibliography style and polishing your final reference list.
\subsubsection{Sourcing BibTeX Files}
When you are looking for different papers, you will most likely use or end up on certain databases, such as \href{https://ieeexplore.ieee.org/Xplore/home.jsp}{IEEE Xplore}, \href{https://www.semanticscholar.org/}{SemanticScholar}, \href{https://www.researchgate.net/}{ResearchGate}, or \href{https://dl.acm.org/}{ACM}. On most databases, including those mentioned, there is an option to export the citation. Most of the time, this can also be in a BibTeX-format. ATTENTION HERE: There are sometimes mistakes in those, you need to double check the results you get! What is also really important is to find the conference/journal/book where something was published. Sometimes, you will find papers on \href{https://arxiv.org/}{arXiv}, which is an open access archive, where papers can be published before they are published through, for example, a journal. You will always need to check, whether this paper has actually been published elsewhere, because only the published reference is correct. \\You can also write a BibTeX entry from scratch directly into the .bib-file, e.g. using \texttt{@article}, which then automatically lists you the most relevant entry-options.
\subsubsection{BibTeX Entry Into Your Project}
If you have a copied BibTeX entry from a database, insert it into your references.bib file. Check it again then! Do you want to include the abstract, are the authors correct, is the abbreviation of your conference consistent with your other references etc. \\\\
To input the bib-file into your project, you need to include the line \texttt{\textbackslash bibliography\{references\}} in your main.tex file right at the end of the document. \textit{References} is the name of your bib-file. This is already done for you in the \texttt{main.tex} file. \\\\
Technically, you can also write the BibTeX-entries into your main.tex but this is really not recommended as it gets messy very quickly.

\subsubsection{Choosing a Bibliography Style}
You always need to define the bibliography style (\texttt{\textbackslash bibliographystyle\{ieeetr\}}) right above the \texttt{\textbackslash bibliography\{references\}} line. 
For this seminar paper, we want you to use the IEEE reference style, which is very common in the Computer Science community. This can be done using, for example, the ieeetr or IEEEtran style. Please have a look, which of those is best for you. I also want to note that there is the option to modify the bibliography style as well (e.g. that more than six authors automatically get displayed as et al.). In my experience this has worked well with \href{https://github.com/maxkratz/ieeetran_doi}{IEEEtranDOI}, where you download the style-file from github and make some changes in there to fit your needs (this enables DOIs to be shown, for example). Note that there is no perfect out-of-the-box solution, you will always need to check and recheck your references for consistency.

\subsubsection{Citing in Your Text}
The references in your bib-file only get displayed when you have used them in the text. You therefore need to \textbf{cite them} using \texttt{\textbackslash cite\{citation-key\}}. If you want to use a quote in your text, please cite it as follows: \\
\texttt{\textbackslash cite[p.5]\{citation-key\}}, which results in something like: \\\\
``Extensive research has established a strong correlation between plagiarism and academic failure, with students who engage in plagiarism consistently achieving the poorest performance in individual examinations'' \cite[p. 1]{Berrezueta-Guzman2023PlagiarismDetection}. \\\\
When you are using quotes, one counterintuitive aspect is that the usual \textbf{double-quotation marks} \texttt{"} don't work correctly in LaTeX. You need to use \texttt{` + `} at the beginning of your quote and \texttt{' + '} at the end. It is better explained in \autoref{fig:quotation-marks}. 
\begin{figure}[h!]
    \centering
    \includegraphics[width=\linewidth]{graphics/Quotation_marks.png}
    \caption{Different quotation marks lead to different results in the final typeset using LaTeX (taken from \cite{Musick0HabitsLatex}).}
    \label{fig:quotation-marks}
\end{figure}
\\
I really recommend that you adapt the \textbf{citation keys} directly after importing your BibTeX entry or generate meaningful entries from the beginning. One example is \texttt{LastnameoffirstauthorYearTitlekeywords}, which could look like: \texttt{Berrezueta-Guzman2023PlagiarismDetection}. Then you have multiple ways to find the correct reference, whether you just remember the last name, something of the title, or need to check against the year. \\\\
Citations should be placed within the sentence, before the full stop, and after a space [1]. When mentioning authors in the text, cite immediately after (e.g., Gelautz et al. \cite{Gelautz2023CVDrivingRobotics} found…).
\subsection{Common Mistakes in the Reference List}
In this section, I will list common mistakes when it comes to the IEEE referencing style:
\begin{itemize}
    \item \textbf{Authors:} This is the IEEE author formatting (Firstname Middlename Lastname)
    \begin{itemize}
        \item One author: F. M. Lastname, ``...."
        \item Two authors: F. M. Lastname and F. M. Lastname, ``..."
        \item Three to six authors: F. M. Lastname, F. M. Lastname, and F. M. Lastname, ``..."
        \item More than six authors: F. M. Lastname et al., ``..."
    \end{itemize}
    \item \textbf{Titles:} Titles of individual articles/papers/reports/chapters are written in double quotation marks and in lowercase except for the first word, proper nouns and after a colon, question mark etc. 
    \item \textbf{Conferences/Journals:} Titles of conferences/books/journals are written in italics and all major words are capitalized. Use abbreviations either on all or none of the conferences (e.g., CVPR)
    \item \textbf{Missing Conferences/Journals:} As previously noted, you should always cite the published version (e.g., via CVPR) over the unpublished version (e.g., via arXiv). But if it was only published via arXiv, you need to mention this as well! Also if it is a diploma thesis/dissertation.
    \item \textbf{Months:} It is not necessary in IEEE to include the month, however if you do, be consistent about which abbreviations you are using or none at all.
\end{itemize}
% \section{Some More Tips}
\textbf{Title Capitalization: } \\ There are \href{https://en.wikipedia.org/wiki/Letter_case#Stylistic_or_specialised_usage}{two main options} for title capitalization: either title case, where you capitalize every significant word, or sentence case, where you capitalize mostly only the first word. Choose one and stay consistent with your titles. \\\\
\textbf{Word Count:} \\
For the seminar papers, you should write about 3500 to 4000 words. If you are using Overleaf, there is an integrated word counter. You can find it, if you go to \texttt{Menu} on the upper left side and then \texttt{Word Count}. In this document, the parts that should not be included for counting are already marked with \texttt{\%TC:ignore} at the beginning and \texttt{\%TC:endignore} at the end. \\\\
\textbf{Literature Management:} I will just give a hint that a literature management tool, such as \href{https://www.zotero.org/}{Zotero} can be very helpful for a seminar paper like this but also for a future thesis. This allows you to have a place where all your literature is saved. You can annotate within the program, create folders, tags, notes, etc. There is even a plugin for most browsers, which allows you to include it directly from where you found it (pay attention to the author, year, conference information here as well!). Zotero is free and open source, only the synchronization of the files between different devices is not unlimited. \\\\
\textbf{Overleaf Professional:} The professional version of Overleaf (online LaTeX editor) is covered by the \href{https://de.overleaf.com/edu/tuw}{TU Wien}! Most things also work with the basic functionality, but it can be a nice add-on to have the professional version. 
%TC:endignore
% ######################################################
% ## CAN ADD OTHER SUPPLEMENTAL MATERIAL AS ANOTHER APPENDIX (THEN "APPENDICES" NOT "APPENDIX") ##

\newpage
%TC:ignore
\addcontentsline{toc}{section}{Appendix}
\section*{Overview of Generative AI Tools Used}
\textbf{AI-Tool:} GitHub Copilot via VSCode\\
\textbf{Purpose:} Grammar correction and sentence completion/rewrite. \\
\textbf{Input:} The LaTeX files from this document are used as context for the model. \\
\textbf{Output:} - \\
\textbf{Reflection:} I had GitHub Copilot enabled as I usually do for coding, writing technical documentation etc. While it started being useful I noticed fast that it tends to write long sentences and you have to be careful to not change the original meaning. Also while it sounded professional it was obvious that the content wasn't self written. Therefore, I stopped about halfway through and completely disabled the extension. \\

\noindent\rule[7pt]{\linewidth}{0.4pt}
\textbf{AI-Tool:} GitHub Copilot via VSCode (Claude Haiku 4.5)\\
\textbf{Purpose:} Creation of an draft of both of the Comparision Tables\\
\textbf{Input:} The LaTeX files from this document are used as context for the model. Along with simple prompts like 'Check my described methods in the 2. chapter methods and update my table in Comparision.tex' \\
\textbf{Output:} Latex Table\\
\textbf{Reflection:} As creation of tables in LaTeX is pretty verbose it saved some time, as it handles columns, formats, size and crossreferences. The actual text content in the first table was changed manually by myself. For the second table I first wrote down the benchmark results from the papers as a latex comment, and let it autofill from the AI into the table, then I verified the changes manually.\\


\noindent\rule[7pt]{\linewidth}{0.4pt}
\textbf{AI-Tool:} GitHub Copilot via VSCode (Claude Haiku 4.5)\\
\textbf{Purpose:} Creation of an draft for the abstract\\
\textbf{Input:} The LaTeX files from this document are used as context for the model. Along with simple prompt like 'Can you write me a draft for my abstract' \\
\textbf{Output:} Draft of the abstract\\
\textbf{Reflection:} Served as a good start. Just summarized the content did not start to get the readers' attention by trying to engage them.\\


% Note: You can display your contents how ever you like, e.g. via a table etc.
%TC:endignore

% ######### REFERENCE PAGE AUTOMATICALLY GENERATED FROM .BIB FILE AND IN TEXT CITATIONS

\newpage
\addcontentsline{toc}{section}{References}
\bibliographystyle{ieeetr}
\bibliography{references}

\end{document}
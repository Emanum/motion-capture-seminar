% This is a template for seminar papers in the courses by Prof. Margrit Gelautz. There are important informations within this document, so please have a look at it at the beginning thorougly! When "TODO" is written, it means that there are changes you need to do here. When "COMMENT" is written, it means that we are trying to explain the LaTeX code for you.

\documentclass[titlepage, 12pt]{article} % The document gets initialized.
\newcommand{\open}[1]{\textcolor{red}{#1}} % When you write: \open{} the text in brackets is red.

% COMMENT: Useful packages
\usepackage[english]{babel} % Language setting
\usepackage[a4paper,top=3cm,bottom=3cm,left=4cm,right=4cm,marginparwidth=1.75cm]{geometry} % Set page size and margins
\usepackage{amsmath, graphicx, gobble, array, xcolor, soul}
\usepackage[colorlinks=true, allcolors=blue]{hyperref} % Creates colored links
\setlength\parindent{0pt} % This removes the indents

% TODO: Please fill in your information here
\newcommand{\paperTitle}{Cost effective Real Time Motion Capture for 3D Avatars}
\newcommand{\studentName}{Manuel Eiweck}
\newcommand{\matriculationNumber}{01633012}
\newcommand{\studyCode}{\open{000 000}}
\newcommand{\studentEmail}{manuel.eiweck@tuwien.ac.at}
\newcommand{\seminarNumber}{193.179}
\newcommand{\seminarName}{Seminar in Visual Computing}
\newcommand{\supervisor}{Prof. Dr. Margrit Gelautz}
\newcommand{\semester}{2025W}
\newcommand{\dueDate}{Draft: 5.12.2025 \\ Final: 5.1.2026}

% COMMENT: This is where your title page is formatted, don't change anything here.
\title{
\textbf{\paperTitle}}
\author{\\ 
    \textbf{\studentName} \\ \\ \\ 
    Matriculation Number: \matriculationNumber \\ 
    Study Code: \studyCode \\ 
    E-mail: \studentEmail \\ \\ 
    \seminarNumber \\
    \seminarName \\
    Supervisor(s): \supervisor \\
    Semester: \semester \\
    \newline
}
\date{\dueDate}

%TC:ignore
% #########################################
% COMMENT: The actual document and content starts here. 
\begin{document}
\maketitle

% #########################################
% COMMENT: The declaration of independent work is inserted here. Please read it again.
\section*{Erklärung zur Verfassung der Arbeit}
Hiermit erkläre ich, dass ich diese Arbeit selbständig verfasst habe, dass ich die verwendeten Quellen und Hilfsmittel vollständig angegeben habe und dass ich die Stellen der Arbeit - einschließlich Tabellen, Karten und Abbildungen -, die anderen Werken oder dem Internet im Wortlaut oder dem Sinn nach entnommen sind, auf jeden Fall unter Angabe der Quelle als Entlehnung kenntlich gemacht habe. \\
\open{Ich erkläre weiters, dass ich mich generativer KI-Tools lediglich als Hilfmittel bedient habe und in der vorliegenden Arbeit meine gestalterischer Einfluss überwiegt. Im Anhang “Übersicht verwendeter Hilfsmittel” habe ich alle generativen KI-Tools gelistet, die verwendet wurden, und angegeben, wo und wie sie verwendet wurden. Für Textpassagen, die ohne substantielle Änderung übernommen wurden, habe ich jeweils die von mir formulierten Eingaben (Prompts) und die verwendete IT-Anwendung mit ihrem Produktnamen und Versionsnummer/Datum angegeben.} \\\\
\open{IMPORTANT: Read this thoroughly so that you make no mistakes here! (it is in German for legal reasons, but you'll find a way to translate it)} % TODO: Remove this before the final submission.

% #########################################
% TODO: Here you should write your abstract.

\newpage
\addcontentsline{toc}{section}{Abstract} % This adds Abstract to the Table of Contents. 
\section*{Abstract}
Here is where your abstract should be. It is not an easy task to write a good abstract, so use some time to think about it. Abstracts shouldn't contain anything that you have not mentioned in the rest of your seminar paper, but should give a short summary about everything that you wrote, including your results. Maybe it helps to think about an abstract as a spoiler: it should give all the information about your paper at the beginning. Also don't confuse it with the role of the introduction. In the introduction, you should motivate the reader thoroughly, give context to the scientific landscape (cite often!) and give an overview of what to expect.
\\\\
Also consider writing your abstract at the end, as you only then fully know what your paper contains.

% #########################################
% COMMENT: This is where your table of content is automatically generated.

\newpage
\tableofcontents
%TC:endignore

% #########################################
% COMMENT: Here, your real content of the seminar paper starts.

\newpage
\pagenumbering{arabic}

% COMMENT: We recommend to create separate files for the separate sections. It simplifies the main.tex file and makes it easier to keep a good overview. You can use \input to include the content of the other file on the same page. Or you can use \include if you want to start a new page with the content.

\section*{Introduction to this Document}
Hello all. This document is supposed to offer you some insights into LaTeX, scientific writing, and give some tips for writing your seminar paper. The content is specifically catered for this seminar and not just generic LaTeX information. If you are looking for a more basic introduction, you can have a look at this \href{https://www.overleaf.com/learn/latex/Learn_LaTeX_in_30_minutes}{LaTeX tutorial}. Please read through the \textbf{whole document thoroughly}, because it focuses on typical mistakes of previous semesters. Also read through the \textbf{introductory slides} again. Please also note: the following is based on my personal experience, can contain mistakes or accidental misinformation.
\\\\
Concerning the template, you don't need to use this template specifically. You are also free to use a different one, but then pay attention that you include all the important aspects which are mentioned in the preliminary slides.

\section{Introduction}
Use the introduction to build a clear storyline that highlights the importance of the topic and engages the reader. Also, give a short overview of what the reader can expect. You should have a lot of references here, as you are laying the scientific basis for the rest of the paper.

\hfill

% While we see professional VTuber Concerts being popular in Asia, we also saw some increasing popularity in the west as well. 
% While big studios use professional grade motion capture systems, similar to these used in the movie business, there are also smaller groups. 
% We believe that with the recent trends and methods of deep learning motion capture systems, more compact systems that require less camera are capable of achieving similar results which are good enough for these use cases. 

% \hfill

Human motion capture (mocap) is heavenly used in the movie industry for computer generated imagery (CGI), dating back all the way to an animation technique called rotorscoping in 1915 by the cartoonist Max Fleischer. While the implementation of these techniques got modernized from a projected movie on a piece of paper to camera and sensor based systems, the goal stayed the same: Capturing real world human motion and translating it into a virtual representation used for animation. \cite{menache_understanding_2000} \\
To achieve as much accuracy as possible, professional state-of-the-art motion capturing studios used a large quantity of high resolution and high frame rate camera systems from multiple perspectives. COVER Corporation uses 200 Vicon Valkyrie (VK-26) \cite{hololive_setup} for their motion capture studio targeted towards computer animation. 
Besides industry applications, generation ground truth datasets also requires high accuracy and therefore apply similar setups. HumanEva \cite{sigal2009humaneva-8bd} build their capture setup using 4 cameras from Vicon MX. MoVi \cite{10.1371/journal.pone.0253157} combines 15 stationary Qualisys cameras, two handheld phone cameras and 17 IMU sensors attached to a bodysuit.
More modern datasets targeted towards deep learning approaches such as the HumanOLAT \cite{teufelgera2025HumanOLAT} use 40 RED Komodo 6K cameras and 331 individually controllable LEDs for a full 360 degree capture dome. 


\hfill 

The taxonomy of high accurate multi-view motion capture systems usually apply fundamentally computer vision algorithms that are already well studied.
A first step is tracking which can be done with model-based, region-based or feature-based methods \cite{wang_recent_2003}. Afterwards, multi-view geometry is applied to do 3D pose estimation. Triangulation is used to calculate the 3D position based on the matching 2D position from the tracked points as well as intrinsic and extrinsic camera parameters \cite{desmarais_review_2021}.
One key challenge in motion capturing is occlusion. This is the reason why many state-of-the-art motion capture systems use a large amount of cameras. As a simple yet effective solution to deal with occlusion is eliminating as much of it in the first place by using multiple camera angles. 
\\
While this produces accurate results it also poses some significant downsides. 
High costs as well as a time intensive setup and calibration process being one of them. 
In addition, recent trends in virtual reality (VR) and augmented reality (AR) head-mounted devices (HMDs) require egocentric motion capture systems that are mobile and work in dynamic environments.
Therefore, we have also seen inside out tracking approaches such as FLAG \cite{FLAG} or EgoPoser \cite{jiang2024egoposer}.
In addition, most approaches presented nowadays are heavenly deep learning based. 

% While there is no clear best approach, many systems used in the wild have in common that they apply classical fundamental computer vision algorithms often in combination with a marker based suit.
% Tracking is being done via model-based, region-based or feature-based tracking methods \cite{wang_recent_2003}. 
% After tracking 
% Multi-view geometry is used for 3D pose estimation. Triangulation is used to translate 2D coordiates to 3D coordinates. This requires 


%  Geometric information
% When several cameras are available, multi-view geometry is frequently used for 3D pose estimation. One way to infer joint coordinates in three dimensions is to use triangulation with their 2D image coordinates in each view. Depending on the calibration and availability of camera extrinsic and intrinsic parameters, different reconstruction schemes are possible.

% The usage of a high number of cameras for classical motion capture methods
% Using a high number of cameras in the first place is a practical way to address the challenge of occlusion. \cite{wang_recent_2003}

\hfill
Relevant Papers:
\\

From Methods to Applications: A Review of Deep
3D Human Motion Capture \cite{FromMethodstoApplicationsAReviewofDeep3DHumanMotionCapture}
\\

Motion Capture Technology in Industrial Applications: A Systematic Review
\cite{McInIndustryASystematicReview}
\section{Task Description}

Describe the Task that needs to be solved in general. 
Realtime, Occlusion, Low cost

Here also whats the general method and steps. 

Also pre deep learning and now


\subsection{Other Usecases}

VR Tracking in general. Inside Out, Outside IN


\subsection{Classical High Cost Motion Capture}

Describe Cinema Grade Motion capture Studios …
How they solve these 

Vicon Valkyrie | Advanced Motion Capture Cameras
\section{Methods}
Keywords:
 monocular video, kinematics, global coordinates, dynamic cameras,  infills missing poses,

\subsection{Old but good}
OpenPose: Realtime Multi-Person 2D Pose Estimation Using Part Affinity Fields \cite{openPose}


XNect: real-time multi-person 3D motion capture with a single RGB camera \cite{mehta_xnect_2020}

\subsection{Non Real Time Approaches}
PACE: Human and Camera Motion Estimation from in-the-wild Videos \cite{PACE}

\subsection{Not categorized yet}
SimPoE: Simulated Character Control for 3D Human Pose Estimation \cite{simpoe}

VIBE: Video Inference for Human Body Pose and Shape Estimation \cite{VIBE}

Avatarpose \cite{leonardis_avatarpose_2025}


\subsection{Datasets}
AMASS: Archive of Motion Capture as Surface Shapes \cite{AMASS:ICCV:2019}

CMU Panoptic Studio datase

Human3.6M, HumanEva, and MPI-INF-3DHP dataset

\pagebreak

\subsection{OpenPose}

OpenPose is a popular open-source framework for real-time 2D multi-person and 3D single-person pose estimation with over 30K stars on GitHub.
There are multiple publications related to OpenPose. Convolutional Pose Machines \cite{openPose_4_wei2016cpm} started with a sequential architecture of convolutional neural networks that produces 2D belief map.
Each network describes a stage, the output of one stage is used as input for the next stage. By using a fully differentiable model they can use backpropagation for training. One issue with their approach is handling multiple people in the same image.\\
This is addressed in their next publication with Part Affinity Fields \cite{openPose_3_cao2017realtime}. A bottom-up instead of top-down approach is used. This means that first body parts are detected and afterwards assigned to an unknown number of people in the image.
They extend their sequential architecture with a two-branch approach. One branch predicts part affinity fields which encode position and orientation of limbs in a 2D vector field, another the confidence maps of body parts at a certain location. By using techniques from graph theory, they can match these body parts to the vector field and also combine them into a full body pose. With their architecture and greedy matching algorithm, they can achieve real-time multi-person 2D pose estimation. \\
In \cite{openPose_2_simon2017hand}, they propose a method that produces 3D motion capture results handling complex Occlusion scenarios. They build up on their previous models to predict 2D estimation via multiple camera angles. The result is then triangulated to get 3D results. 
However, compared to other approaches they go a step further and use these results as their training data together with their 2D estimations for a model which enables markerless 3D motion capture outputs on hands from single view RGB images.
In \cite{openPose}, they extend their 2D multi-person pose estimation to also include foot keypoints and facial landmarks, compared to their previous wor they only use Part Affinity Fields. Furthermore, this work also does not include any 3D motion capture cababilities. In their GitHub \cite{openPose_github} they provide an option to detect 3D keypoints including face, hand and foot features using multiple cameras. However, as it uses simple triangulation it has the limitation of a fixed camera setup and only works with a single person in the scene.

\subsection{DeMoCap}
Chatzitofis et al. proposed their 3D marker based motion capture system DeMoCap \cite{chatzitofis2021democap}. Their work includes the method itself, focused on providing a low-cost alternative to the classical optical marker based solutions by using consumer-grade infrared-depth camera only. In addition, they release the dataset used for training, which contains colored infrared and depth images with 3D pose and marker annotations in multiple views. The ground truth data comes from professional motion capture system with 24 cameras while the depth data comes from 4 stereo based depth sensors.\\
Their proposed method works as follows: First for each of the multi-view depth images 3D positions of the markers are extracted and normalized. This work as each marker reflects the infrared rays with a different intensity therefore identifying the marker. 
These markers together with joint heatmaps are fed into a fully convolutional network that transforms the markers to poses. This is done via their own 3D regression model. \\
While their methods performs quiet good even under complex scenarios on public datasets such as the SFU Dataset \cite{sfudataset} it comes with some serious limitations. 
Their used depth sensors have a distance limitation of 4 meters, which limits the capture space significantly especially compared to optical sensor based systems. Furthermore, the quantity of fast poses are challenging due to the low 30hz frequency of the sensor. 
With only 2-4 cameras required the system is fairly mobile and quickly to set up. However, it requires the actor to wear a special suit with 53 placed markers which make it far from ideal for applications in the wild.

\subsection{XNect}

Xnect \cite{mehta_xnect_2020} is a real-time markerless approach that is powered by a single monocular RGB based camera and can provide temporally coherent tracing tracking for multiple people in diverse scenes in the wild.

Their bottom up architecture start by applying a convolutional neural network which is trained to detect only fully visible features such as the joint itself, or it's parent/child. The networks output are 2D body joint heatmaps, part affinity fields to assign joints to individual persons as well as their own 3D pose encoder which uses local pixel information close to the 2D joint position. 
In a second step a lightweight fully connected neural network transform the 2D features and 3D encoder to 3D poses for each person. In addition, it uses previous learned 3D poses to handle occlusion within the scene. 
Lastly, a space-time skeletal model is applied to further improve the 3D pose. 
The result or the last two stages is used in combination with the results of previous frames to fit the 3D pose to a kinematic skeleton and therefore enforce temporal coherence.
The end result of the pipeline is a full skeletal pose with joint angles for each person.

While their CNN applies a novel CNN architecture called SelecSLS to achieve real-time performance, they mention that any CNN architecture for keypoint prediction can be used.
This makes their approach potentially be extended in the future by upcoming CNN network architectures. 

They compare their approach to Microsoft Kinect V2 which uses depth sensing cameras with similar and sometimes better qualitative performance especially for scenes including occlusions. 
In addition, results on common datasets such as MuPoTS-3D benchmark dataset (TODO cite) and the Panoptic dataset (TODO cite) are presented and compared to SingleShot [Mehta et al. 2018b], LCRNet [Rogez et al. 2017], MP3D [Dabral et al. 2019], LCRNet++ [Rogez et al. 2019] and PoseNet (https://arxiv.org/abs/1505.07427). The only approach that had a better score was PoseNet, however it comes with other limitations one of which is the lack of real-time performance. Their own CNN SelecSLS is compared to ResNet (TODO cite) with similar results while still also lacking real-time cababilities.
However, compared to multi-view capture systems XNect has far less accuracy. 
While occulusion is partly handled via their 2nd stage, it is far from perfect. 
Occlusion of the neck of a person result in the entire person not being detected regarding if large parts of the body is detected. Close human interactions like hugging also leads to inaccurate results. 
Using previous frames to match a skeleton can lean to issues for the first frames of occlusions. The simple person tracker also lead to mismatches between the capture and therefore a swap.


\subsection{MAMMA}


\section{Motion Capture in Industry}

SMPL: A Skinned Multi-Person Linear Model

SMPL-X format. 

Skeleton


\cite{loper_smpl_2015}

VMC Protocol Specification \cite{vmc_protocol_specification}

VRM File Format \cite{VRM_consortium}

VRM Press Release \cite{vrm_press_release}

Sony Mocap Protocol https://www.sony.co.jp/en/Products/mocopi-dev/en/documents/Home/TechSpec.html

Webcam Motion Capture + Sony Mocap
https://www.sony.co.jp/en/Products/mocopi-dev/en/documents/Home/TechSpec.html


\subsection{Datasets}
AMASS: Archive of Motion Capture as Surface Shapes \cite{AMASS:ICCV:2019}

CMU Panoptic Studio datase

Human3.6M, HumanEva, and MPI-INF-3DHP dataset


DanceDB (included in AMASS)


\section{Discussion}

While you don't have to include a discussion section, it is encouraged that you try to add to existing knowledge in some way. This may be through comparison of existing methods, critical reflection, etc.

\section{Conclusion}
Conclusion goes here. These were some of the most important aspects of the seminar paper. If you have any more questions, feel free to reach out to the seminar supervisors. 


\section{Short Introduction into LaTeX}
Always include a small introductory section before you start the next subsection. Also, if you only have one subsection, leave it out. In this case: In this section, you can find information about figures, BibTeX entries (including databases, bibliography style, and citations), and common mistakes to avoid.

\subsection{Inserting Figures}
You can and should add figures to your seminar paper. This works using  \textit{\textbackslash begin\{figure\}}. There are five things to look out for here:
\begin{enumerate}
    \item You can add a [h!] after \texttt{\textbackslash begin\{figure\}}. This helps to position the figure in the text, where you mentioned it (though this is not 100\%).
    \item You should give a meaningful and detailed description of the figure in the caption. You can do this in the \texttt{caption} field of the \texttt{\textbackslash begin\{figure\}} (see \autoref{fig:color-blind-chart} for an example).
    \item Make sure that you reference where you have the image from. If you created it yourself, you can but don't necessarily need to mention this. If you adapted it from somewhere, note that as well.
    \item Your figure should be readable and accessible. This means that the text in the figure/image should be approximately the same font size as the rest of your text. Also, have a look for the colors, they should be readable for color-blind folks as well. I can recommend this tool to check for that: \href{https://www.color-blindness.com/coblis-color-blindness-simulator/}{Color Blindness Simulator}.\\
    \begin{figure}[h!]
        \centering
        \includegraphics[width=\linewidth]{graphics/Color_blindness_wheels.png}
        \caption{Three different color blindness types are presented with the standard vision (Trichromacy) in comparison. These include Deuteranopia (green-blind), Protanopia (red-blind), and Trianopia (blue-blind) (adapted from \cite{Hawesthoughts2023Colorblindness}).}
        \label{fig:color-blind-chart}
    \end{figure}
    \item And lastly, reference the figure in the text using \texttt{ref\{\}}. Ideally, you discuss the figure in the surrounding text to integrate it in the structure and justify its inclusion in the seminar paper. Like so: \textit{A visualization of common color blindness types can be found in Figure~\ref{fig:color-blind-chart}. Most common is Deuteranopia, which makes the distinction between red and green especially difficult (compare to Figure \ref{fig:color-blind-chart}).}
    You can also use \texttt{autoref\{\}} which look like this: \autoref{fig:color-blind-chart}. 
\end{enumerate}

\subsection{Citations in LaTeX}
If you've worked with LaTeX before, you are probably familiar with the following information. In LaTeX, it is common that you have a separate bib-file (here: \texttt{references.bib}), which contains your BibTeX entries. A BibTeX entry contains all relevant information about a reference. There are different entry-types, such as \texttt{article}, \texttt{book}, etc. In the following, you will get information about sourcing BibTeX entries, inserting them into your project, choosing a bibliography style and polishing your final reference list.
\subsubsection{Sourcing BibTeX Files}
When you are looking for different papers, you will most likely use or end up on certain databases, such as \href{https://ieeexplore.ieee.org/Xplore/home.jsp}{IEEE Xplore}, \href{https://www.semanticscholar.org/}{SemanticScholar}, \href{https://www.researchgate.net/}{ResearchGate}, or \href{https://dl.acm.org/}{ACM}. On most databases, including those mentioned, there is an option to export the citation. Most of the time, this can also be in a BibTeX-format. ATTENTION HERE: There are sometimes mistakes in those, you need to double check the results you get! What is also really important is to find the conference/journal/book where something was published. Sometimes, you will find papers on \href{https://arxiv.org/}{arXiv}, which is an open access archive, where papers can be published before they are published through, for example, a journal. You will always need to check, whether this paper has actually been published elsewhere, because only the published reference is correct. \\You can also write a BibTeX entry from scratch directly into the .bib-file, e.g. using \texttt{@article}, which then automatically lists you the most relevant entry-options.
\subsubsection{BibTeX Entry Into Your Project}
If you have a copied BibTeX entry from a database, insert it into your references.bib file. Check it again then! Do you want to include the abstract, are the authors correct, is the abbreviation of your conference consistent with your other references etc. \\\\
To input the bib-file into your project, you need to include the line \texttt{\textbackslash bibliography\{references\}} in your main.tex file right at the end of the document. \textit{References} is the name of your bib-file. This is already done for you in the \texttt{main.tex} file. \\\\
Technically, you can also write the BibTeX-entries into your main.tex but this is really not recommended as it gets messy very quickly.

\subsubsection{Choosing a Bibliography Style}
You always need to define the bibliography style (\texttt{\textbackslash bibliographystyle\{ieeetr\}}) right above the \texttt{\textbackslash bibliography\{references\}} line. 
For this seminar paper, we want you to use the IEEE reference style, which is very common in the Computer Science community. This can be done using, for example, the ieeetr or IEEEtran style. Please have a look, which of those is best for you. I also want to note that there is the option to modify the bibliography style as well (e.g. that more than six authors automatically get displayed as et al.). In my experience this has worked well with \href{https://github.com/maxkratz/ieeetran_doi}{IEEEtranDOI}, where you download the style-file from github and make some changes in there to fit your needs (this enables DOIs to be shown, for example). Note that there is no perfect out-of-the-box solution, you will always need to check and recheck your references for consistency.

\subsubsection{Citing in Your Text}
The references in your bib-file only get displayed when you have used them in the text. You therefore need to \textbf{cite them} using \texttt{\textbackslash cite\{citation-key\}}. If you want to use a quote in your text, please cite it as follows: \\
\texttt{\textbackslash cite[p.5]\{citation-key\}}, which results in something like: \\\\
``Extensive research has established a strong correlation between plagiarism and academic failure, with students who engage in plagiarism consistently achieving the poorest performance in individual examinations'' \cite[p. 1]{Berrezueta-Guzman2023PlagiarismDetection}. \\\\
When you are using quotes, one counterintuitive aspect is that the usual \textbf{double-quotation marks} \texttt{"} don't work correctly in LaTeX. You need to use \texttt{` + `} at the beginning of your quote and \texttt{' + '} at the end. It is better explained in \autoref{fig:quotation-marks}. 
\begin{figure}[h!]
    \centering
    \includegraphics[width=\linewidth]{graphics/Quotation_marks.png}
    \caption{Different quotation marks lead to different results in the final typeset using LaTeX (taken from \cite{Musick0HabitsLatex}).}
    \label{fig:quotation-marks}
\end{figure}
\\
I really recommend that you adapt the \textbf{citation keys} directly after importing your BibTeX entry or generate meaningful entries from the beginning. One example is \texttt{LastnameoffirstauthorYearTitlekeywords}, which could look like: \texttt{Berrezueta-Guzman2023PlagiarismDetection}. Then you have multiple ways to find the correct reference, whether you just remember the last name, something of the title, or need to check against the year. \\\\
Citations should be placed within the sentence, before the full stop, and after a space [1]. When mentioning authors in the text, cite immediately after (e.g., Gelautz et al. \cite{Gelautz2023CVDrivingRobotics} found…).
\subsection{Common Mistakes in the Reference List}
In this section, I will list common mistakes when it comes to the IEEE referencing style:
\begin{itemize}
    \item \textbf{Authors:} This is the IEEE author formatting (Firstname Middlename Lastname)
    \begin{itemize}
        \item One author: F. M. Lastname, ``...."
        \item Two authors: F. M. Lastname and F. M. Lastname, ``..."
        \item Three to six authors: F. M. Lastname, F. M. Lastname, and F. M. Lastname, ``..."
        \item More than six authors: F. M. Lastname et al., ``..."
    \end{itemize}
    \item \textbf{Titles:} Titles of individual articles/papers/reports/chapters are written in double quotation marks and in lowercase except for the first word, proper nouns and after a colon, question mark etc. 
    \item \textbf{Conferences/Journals:} Titles of conferences/books/journals are written in italics and all major words are capitalized. Use abbreviations either on all or none of the conferences (e.g., CVPR)
    \item \textbf{Missing Conferences/Journals:} As previously noted, you should always cite the published version (e.g., via CVPR) over the unpublished version (e.g., via arXiv). But if it was only published via arXiv, you need to mention this as well! Also if it is a diploma thesis/dissertation.
    \item \textbf{Months:} It is not necessary in IEEE to include the month, however if you do, be consistent about which abbreviations you are using or none at all.
\end{itemize}
\section{Some More Tips}
\textbf{Title Capitalization: } \\ There are \href{https://en.wikipedia.org/wiki/Letter_case#Stylistic_or_specialised_usage}{two main options} for title capitalization: either title case, where you capitalize every significant word, or sentence case, where you capitalize mostly only the first word. Choose one and stay consistent with your titles. \\\\
\textbf{Word Count:} \\
For the seminar papers, you should write about 3500 to 4000 words. If you are using Overleaf, there is an integrated word counter. You can find it, if you go to \texttt{Menu} on the upper left side and then \texttt{Word Count}. In this document, the parts that should not be included for counting are already marked with \texttt{\%TC:ignore} at the beginning and \texttt{\%TC:endignore} at the end. \\\\
\textbf{Literature Management:} I will just give a hint that a literature management tool, such as \href{https://www.zotero.org/}{Zotero} can be very helpful for a seminar paper like this but also for a future thesis. This allows you to have a place where all your literature is saved. You can annotate within the program, create folders, tags, notes, etc. There is even a plugin for most browsers, which allows you to include it directly from where you found it (pay attention to the author, year, conference information here as well!). Zotero is free and open source, only the synchronization of the files between different devices is not unlimited. \\\\
\textbf{Overleaf Professional:} The professional version of Overleaf (online LaTeX editor) is covered by the \href{https://de.overleaf.com/edu/tuw}{TU Wien}! Most things also work with the basic functionality, but it can be a nice add-on to have the professional version. 

% ######################################################
% ## CAN ADD OTHER SUPPLEMENTAL MATERIAL AS ANOTHER APPENDIX (THEN "APPENDICES" NOT "APPENDIX") ##

\newpage
%TC:ignore
\addcontentsline{toc}{section}{Appendix}
\section*{Overview of Generative AI Tools Used}
\textbf{AI-Tool:} \\
\textbf{Purpose:} \\
\textbf{Input:}\\
\textbf{Output:} \\
\textbf{Usage:} \\
\textbf{Reflection:} \\

\noindent\rule[7pt]{\linewidth}{0.4pt}
\textbf{AI-Tool:} \\
\textbf{Purpose:} \\
\textbf{Input:}\\
\textbf{Output:} \\
\textbf{Usage:} \\
\textbf{Reflection:} \\
\\
Note: You can display your contents how ever you like, e.g. via a table etc.
%TC:endignore

% ######### REFERENCE PAGE AUTOMATICALLY GENERATED FROM .BIB FILE AND IN TEXT CITATIONS

\newpage
\addcontentsline{toc}{section}{References}
\bibliographystyle{ieeetr}
\bibliography{references}

\end{document}
\section{Introduction}
% Use the introduction to build a clear storyline that highlights the importance of the topic and engages the reader. Also, give a short overview of what the reader can expect. You should have a lot of references here, as you are laying the scientific basis for the rest of the paper.

% \hfill

% While we see professional VTuber Concerts being popular in Asia, we also saw some increasing popularity in the west as well. 
% While big studios use professional grade motion capture systems, similar to these used in the movie business, there are also smaller groups. 
% We believe that with the recent trends and methods of deep learning motion capture systems, more compact systems that require less camera are capable of achieving similar results which are good enough for these use cases. 

% \hfill

Human motion capture is heavenly used in the movie industry for computer generated imagery (CGI), dating back all the way to an animation technique called rotorscoping in 1915 by the cartoonist Max Fleischer. While the implementation of these techniques got modernized from a projected movie on a piece of paper to optical sensor based systems, the goal stayed the same: Capturing real world human motion data and translating them into a virtual representation for further processing such as in traditional 2D hand-drawn animations or modern 3D computer generated animations \cite{menache_understanding_2000}. \\
To achieve as much accuracy as possible, traditional state-of-the-art motion capturing studios use a large quantity of high resolution and high frame rate camera systems in a multi view setup combined. This is often referred to as the "gold standard" in literature.
Professional studio equipment as well as proprietary software is developed by companies like Vicon \cite{VICON}, Optitrack \cite{optitrack} or Qualisys \cite{qualisys}.
While smaller areas can be captures using only 4 cameras \cite{sigal2009humaneva-8bd}. Cover corporation demonstrate the scaling possibilities of this technique by equipping a 23 m x 14 m x 3.2 m studio with 200 Vicon Valkyrie (VK-26) \cite{hololive_setup}. 
Similar setups are also applied in the research area of human pose estimation when high accuracy is requires such as the generation of ground truth datasets. HumanEva \cite{sigal2009humaneva-8bd} build their capture setup using 4 cameras from Vicon MX. MoVi \cite{10.1371/journal.pone.0253157} combines 15 stationary Qualisys cameras and 17 IMU sensors.
More modern datasets targeted towards deep learning approaches such as the HumanOLAT \cite{teufelgera2025HumanOLAT} use 40 RED Komodo 6K cameras and 331 individually controllable LEDs for a full 360 degree capture dome. The CMU Panoptic \cite{Joo_2017_TPAMI} uses over 500 cameras together with other sensors.\\
Approaches following the "gold standard" usually apply fundamentally computer vision algorithms that are already well studied. 
A first step is getting marker positions which can be done with model-based, region-based or feature-based methods \cite{wang_recent_2003}. Afterwards, multi-view geometry is applied to do 3D pose estimation. 
During this step triangulation is used to calculate the 3D position based on the matching 2D positions from the tracked points as well as intrinsic and extrinsic camera parameters \cite{desmarais_review_2021}.
One key challenge in motion capturing is occlusion. A simple yet effective solution to deal with this is eliminating as much of it in the first place by using multiple cameras. 
Therefore, many state-of-the-art motion capture systems in the industry as well approaches in research use a large amount of cameras to cover as many angles as possible.

\hfill 

While these methods still deliver accurate results, they also have some significant downsides.
One issue is the requirement for a fixed physical capture area. Recent trends in virtual reality (VR) and augmented reality (AR) created the need for egocentric motion capture systems based on the camera perspective available in head-mounted devices (HMDs). The presence of dynamic moving environments and therefore missing camera calibration making these tasks especially challenging. FLAG \cite{FLAG} and EgoPoser \cite{jiang2024egoposer} apply a transformer based deep learning approach to approach this problem.\\
Another obvious issue is the high hardware cost due to the sheer amount of high resolution camera being used. Increased costs and unflexible setups might not be an issue for professional motion capture studios or the dataset generation in research projects. However, it creates a high entry barrier into motion capture for smaller organizations and individual creatives such as 3D artists, 3D animators or VR creators and musicians performing via a virtual avatar (VTuber). 
Modern approaches often require a significantly lower number of cameras. Additional sensor data such as IMU sensors, deep learning and specialized techniques such as modelling of limbs is used instead \cite{chatzitofis2021democap} \cite{mehta_xnect_2020} \cite{openPose} \cite{VIBE} \cite{simpoe} \cite{leonardis_avatarpose_2025}
\cite{FromMethodstoApplicationsAReviewofDeep3DHumanMotionCapture} \cite{McInIndustryASystematicReview}.

\hfill

% While professional multi-view motion capture systems will continue to provide the most accurate motion capture results,
% can fill in the gap between high accurate multi-view motion capture systems and 

We believe that recent developments in the 3D pose estimation research field can provide motion capture systems for individual creators or smaller companies. 
The focus of these systems target high portability, lower costs, less invasive to the human body and application in dynamic environments compared to traditional systems installed in permanent motion capture studios. We want to explore these recent methods and technologies and evaluate their applicability for these use cases.

\hfill 

The motion capture research field is large and there exist many completely different approaches using different sensors, architectures, requirements and goals in mind. Therefore, in chapter \ref{chapter_methods} we are presenting some popular and current state-of-the-art methods that all approach these challenges using different techniques.
Chapter \ref{chapter_tech} describes datasets, data format as well as protocol and standards used in the industry. 
In \ref{chapter_task} our task is presented which focus around real time motion capture for live concerts with virtual avatars. 
In \ref{chapter_discussion} we categorize and compare approaches. 

% Note to myself:
% DeMoCap \cite{chatzitofis2021democap} has some good infi for intro


% Animate virtual Avatars 

% Optical human motion capture is a key enabling technology in
% visual computing and related fields [Chai and Hodgins 2005; Men-
% ache 2010; Starck and Hilton 2007]. For instance, it is widely used
% to animate virtual avatars and humans in VFX. It is a key compo-
% nent of many man-machine interfaces and is central to biomedical
% motion analysis
% FROM XNECT


% gold-standard
% marker-based solution


% Sony Mocopi case study-> example Usage
% https://pro.sony/ue_US/press/uc-san-diego-spatial-reality-display-mocopi-case-study

% Commercial systems
% https://www.motionanalysis.com/
% https://www.optitrack.com/
% https://www.qualisys.com/

% Company ffrom the max planck institute guy
% https://meshcapade.com/


% AMASS: Archive of Motion Capture as Surface Shape -> discusses why gold standard is ass

% While
% commercial markerless systems exist [Captury 2025; Move AI Ltd.
% 2025; Moverse 2025; Theia Markerless 2025], they remain expensive
% and are generally considered to be less accurate than marker-based
% systems (the “gold standard”)

\section{Problem Statement}
\label{chapter_task}

As we have seen in the introduction, the approaches vary significantly depending on the applied use case. 
We want to group approaches via their intended application areas and give a short overview. 
Afterwards, we will derive our problem statement for this seminar paper and try to categorize it into common taxonomies.



First we define our problem statement

Describe the Task that needs to be solved in general. 
Realtime, Occlusion, Low cost

Here also whats the general method and steps. 

Also pre deep learning and now


\subsection{Other Usecases}

VR Tracking in general. Inside Out, Outside IN


\subsection{Classical High Cost Motion Capture}

Describe Cinema Grade Motion capture Studios …
How they solve these 

Vicon Valkyrie | Advanced Motion Capture Cameras

\subsection{Terms}

join angle result motion capture

and produces per-frame joint position esti-
mates which cannot be directly employed in many end applications
requiring joint angle based avatar animations.
FROM XNECT \cite{mehta_xnect_2020}


\subsection{Comparison of approaches}

Egocentric Pose estimations
FRAME https://www.youtube.com/watch?v=TADocPfpS3s

EgoPoseFormer Yang et Al 2024

UnrealEgo Akada et Al 2022

EgoGlass Zhao et al 2021


Modal based constraints: 


we fit a model-based skeleton to the 3D and 2D predictions in order
to satisfy kinematic constraints and reconcile the 2D and 3D predic-
tions across time -> XNect
